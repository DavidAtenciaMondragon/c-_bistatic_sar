\documentclass[12pt,a4paper]{article}
\usepackage[utf8]{inputenc}
\usepackage[portuguese]{babel}
\usepackage{amsmath}\begin{document}

% ==============================================================================
% TÍTO Princípio de Fermat estabelece que a luz percorre o caminho que minimiza o tempo de percurso. Para um meio estratificado com dois índices de refração, o problema de otimização é formulado como:

\begin{destacadoazul}
\begin{equation}
\boxed{\min_{\mathbf{P}_{\text{refr}}} \mathcal{L}(\mathbf{P}_{\text{refr}}) = n_1 \|\mathbf{P}_{\text{refr}} - \mathbf{T}_x\|_2 + n_2 \|\mathbf{T}_r - \mathbf{P}_{\text{refr}}\|_2}
\label{eq:fermat}
\end{equation}
\end{destacadoazul}

\begin{tcolorbox}[colback=lightgray!20,colframe=darkgray,arc=5pt]
\textbf{Parâmetros do Problema:}
\begin{itemize}
    \item $\mathbf{T}_x \in \mathbb{R}^3$ é a posição do transmissor
    \item $\mathbf{T}_r \in \mathbb{R}^3$ é a posição do alvo
    \item $\mathbf{P}_{\text{refr}} \in \mathbb{R}^3$ é o ponto de refração (variável de otimização)
    \item $n_1, n_2 \in \mathbb{R}^+$ são os índices de refração dos meios
\end{itemize}
\end{tcolorbox}ADO COM DESIGN ELEGANTE
% ==============================================================================
\begin{titlepage}
\centering

% Logo ou símbolo decorativo
\begin{tikzpicture}
\draw[primaryblue, line width=3pt] (0,0) circle (1);
\draw[accentgreen, line width=2pt] (-0.7,-0.7) -- (0.7,0.7);
\draw[accentgreen, line width=2pt] (-0.7,0.7) -- (0.7,-0.7);
\fill[primaryblue] (0,0) circle (0.1);
\end{tikzpicture}

\vspace{1cm}

% Título principal
{\Huge\textcolor{primaryblue}{\textbf{Otimização de Algoritmos de}}}\\[0.3cm]
{\Huge\textcolor{primaryblue}{\textbf{Determinação de Trajetórias Ótimas}}}\\[0.2cm]
{\Large\textcolor{darkgray}{\textbf{em Sistemas de Radar Bistático}}}\\[0.8cm]

% Subtítulo
\begin{tcolorbox}[colback=secondaryblue!20,colframe=primaryblue,width=0.8\textwidth,arc=5pt]
\centering
{\large\textcolor{primaryblue}{\textbf{Uma Abordagem Paralela Baseada no Princípio de Fermat}}}
\end{tcolorbox}

\vspace{2cm}

% Informações dos autores
\begin{destacadoverde}
\centering
{\Large\textbf{Laboratório de Processamento de Sinais de Radar}}\\[0.3cm]
{\large Departamento de Engenharia de Sistemas}\\[0.2cm]
{\normalsize\textcolor{darkgray}{\today}}
\end{destacadoverde}

\vspace{1.5cm}

% Resumo visual dos resultados
\begin{tcolorbox}[colback=lightgray,colframe=darkgray,arc=8pt,width=0.9\textwidth]
\centering
\textcolor{primaryblue}{\textbf{Principais Resultados}}\\[0.2cm]
\begin{minipage}{0.25\textwidth}
\centering
{\Huge\textcolor{accentgreen}{\textbf{4,95×}}}\\
{\small Speedup}
\end{minipage}
\begin{minipage}{0.25\textwidth}
\centering
{\Huge\textcolor{primaryblue}{\textbf{79,8\%}}}\\
{\small Redução Temporal}
\end{minipage}
\begin{minipage}{0.25\textwidth}
\centering
{\Huge\textcolor{warningorange}{\textbf{10⁻⁶}}}\\
{\small Precisão (m)}
\end{minipage}
\end{tcolorbox}

\vfill

\end{titlepage}

% ==============================================================================
% ABSTRACT COM DESIGN ATRATIVO
% ==============================================================================
\newpage
\begin{destacadoazul}
\section*{\textcolor{primaryblue}{Abstract}}
Este trabalho apresenta uma metodologia de otimização computacional para algoritmos de determinação de trajetórias ótimas em sistemas de radar bistático. O problema central consiste na aplicação eficiente do Princípio de Fermat para determinação de caminhos de propagação eletromagnética através de interfaces com descontinuidades de índice de refração. A metodologia proposta emprega paralelização em memória compartilhada e vetorização matemática para processar grids tridimensionais de grande escala. Os resultados demonstram um speedup de 4,95× mantendo precisão numérica equivalente ao método sequencial de referência.
\end{destacadoazul}msfonts}
\usepackage{amssymb}
\usepackage{graphicx}
\usepackage{booktabs}
\usepackage{array}
\usepackage{multirow}
\usepackage{longtable}
\usepackage{geometry}
\usepackage{fancyhdr}
\usepackage{algorithm}
\usepackage{algorithmic}
\usepackage{listings}
\usepackage{xcolor}
\usepackage{float}
\usepackage{siunitx}
\usepackage{titlesec}
\usepackage{tcolorbox}
\usepackage{mdframed}
\usepackage{tikz}
\usepackage{pgfplots}
\usepackage{enumitem}
\usepackage{microtype}
\usepackage[font=small,labelfont=bf]{caption}

% ==============================================================================
% CONFIGURAÇÃO DE CORES ELEGANTES
% ==============================================================================
\definecolor{primaryblue}{RGB}{25,118,210}     % Azul principal
\definecolor{secondaryblue}{RGB}{144,202,249}  % Azul claro
\definecolor{accentgreen}{RGB}{76,175,80}      % Verde destaque
\definecolor{warningorange}{RGB}{255,152,0}    % Laranja aviso
\definecolor{darkgray}{RGB}{66,66,66}          % Cinza escuro
\definecolor{lightgray}{RGB}{245,245,245}      % Cinza claro
\definecolor{mathblue}{RGB}{21,101,192}        % Azul matemática

% ==============================================================================
% CONFIGURAÇÃO DE GEOMETRIA E LAYOUT
% ==============================================================================
\geometry{
    margin=2.5cm,
    headheight=15pt,
    footskip=30pt
}

% ==============================================================================
% CONFIGURAÇÃO DE CABEÇALHOS E RODAPÉS
% ==============================================================================
\pagestyle{fancy}
\fancyhf{}
\fancyhead[L]{\textcolor{primaryblue}{\small\textbf{Otimização em Radar Bistático}}}
\fancyhead[R]{\textcolor{darkgray}{\small\thepage}}
\fancyfoot[C]{\textcolor{darkgray}{\small Laboratório de Processamento de Sinais}}
\renewcommand{\headrulewidth}{1pt}
\renewcommand{\footrulewidth}{0.5pt}
\renewcommand{\headrule}{\hbox to\headwidth{\color{primaryblue}\leaders\hrule height \headrulewidth\hfill}}
\renewcommand{\footrule}{\hbox to\headwidth{\color{lightgray}\leaders\hrule height \footrulewidth\hfill}}

% ==============================================================================
% CONFIGURAÇÃO DE TÍTULOS E SEÇÕES
% ==============================================================================
\titleformat{\section}
{\color{primaryblue}\Large\bfseries}
{\textcolor{primaryblue}{\thesection}}{1em}{}
[\textcolor{primaryblue}{\titlerule[2pt]}]

\titleformat{\subsection}
{\color{darkgray}\large\bfseries}
{\textcolor{primaryblue}{\thesubsection}}{1em}{}

\titleformat{\subsubsection}
{\color{darkgray}\normalsize\bfseries}
{\textcolor{primaryblue}{\thesubsubsection}}{1em}{}

% ==============================================================================
% CONFIGURAÇÃO DE CAIXAS E AMBIENTES DESTACADOS
% ==============================================================================
\newtcolorbox{destacadoazul}{
    colback=secondaryblue!20,
    colframe=primaryblue,
    boxrule=1pt,
    arc=3pt,
    left=10pt,right=10pt,top=10pt,bottom=10pt
}

\newtcolorbox{destacadoverde}{
    colback=accentgreen!15,
    colframe=accentgreen,
    boxrule=1pt,
    arc=3pt,
    left=10pt,right=10pt,top=10pt,bottom=10pt
}

\newtcolorbox{equacaodestaque}{
    colback=mathblue!10,
    colframe=mathblue,
    boxrule=1.5pt,
    arc=5pt,
    left=15pt,right=15pt,top=15pt,bottom=15pt
}

% ==============================================================================
% CONFIGURAÇÃO DE LISTAS
% ==============================================================================
\setlist[itemize,1]{label=\textcolor{primaryblue}{$\bullet$}}
\setlist[itemize,2]{label=\textcolor{accentgreen}{$\circ$}}
\setlist[enumerate,1]{label=\textcolor{primaryblue}{\arabic*.}}

% ==============================================================================
% CONFIGURAÇÃO DE TABELAS
% ==============================================================================
\renewcommand{\arraystretch}{1.3}

% ==============================================================================
% TÍTULO PRINCIPAL COM DESIGN ATRATIVO
% ==============================================================================

\title{\textbf{Otimização de Algoritmos de Determinação de Trajetórias Ótimas \\
em Sistemas de Radar Bistático: \\
Uma Abordagem Paralela Baseada no Princípio de Fermat}}

\author{Laboratório de Processamento de Sinais de Radar\\
Departamento de Engenharia de Sistemas\\
\today}

\begin{document}

\maketitle

\begin{abstract}
Este trabalho apresenta uma metodologia de otimização computacional para algoritmos de determinação de trajetórias ótimas em sistemas de radar bistático. O problema central consiste na aplicação eficiente do Princípio de Fermat para determinação de caminhos de propagação eletromagnética através de interfaces com descontinuidades de índice de refração. A metodologia proposta emprega paralelização em memória compartilhada e vetorização matemática para processar grids tridimensionais de grande escala. Os resultados demonstram um speedup de 4,95× mantendo precisão numérica equivalente ao método sequencial de referência.
\end{abstract}

\tableofcontents
\newpage

%==============================================================================
\section{Introdução}
%==============================================================================

Sistemas de radar bistático requerem a determinação precisa de trajetórias de propagação eletromagnética através de meios com propriedades dielétricas heterogêneas. O problema fundamental consiste na minimização do caminho óptico entre transmissor e receptor, considerando a refração na interface ar-solo, conforme estabelecido pelo Princípio de Fermat.

Para aplicações de imageamento e detecção em larga escala, o processamento de grids tridimensionais densos apresenta desafios computacionais significativos. Um grid típico de dimensões $41 \times 41 \times 21$ pontos, com $M = 6.689$ trajetórias por ponto, resulta em aproximadamente $1,67 \times 10^{11}$ operações de ponto flutuante.

Este trabalho desenvolve e avalia metodologias de otimização computacional baseadas em paralelização e vetorização para acelerar o processamento destes algoritmos, mantendo-se a precisão numérica requerida para aplicações científicas.

%==============================================================================
\section{Formulação Matemática}
%==============================================================================

\subsection{Princípio de Fermat em Meios Estratificados}

O Princípio de Fermat estabelece que a luz percorre o caminho que minimiza o tempo de percurso. Para um meio estratificado com dois índices de refração, o problema de otimização é formulado como:

\begin{equation}
\min_{\mathbf{P}_{\text{refr}}} \mathcal{L}(\mathbf{P}_{\text{refr}}) = n_1 \|\mathbf{P}_{\text{refr}} - \mathbf{T}_x\|_2 + n_2 \|\mathbf{T}_r - \mathbf{P}_{\text{refr}}\|_2
\label{eq:fermat}
\end{equation}

onde:
\begin{itemize}
    \item $\mathbf{T}_x \in \mathbb{R}^3$ é a posição do transmissor
    \item $\mathbf{T}_r \in \mathbb{R}^3$ é a posição do alvo
    \item $\mathbf{P}_{\text{refr}} \in \mathbb{R}^3$ é o ponto de refração (variável de otimização)
    \item $n_1, n_2 \in \mathbb{R}^+$ são os índices de refração dos meios
\end{itemize}

\subsection{Discretização e Método de Solução}

O ponto de refração $\mathbf{P}_{\text{refr}}$ é restrito a uma superfície bidimensional representando a interface ar-solo. A solução é obtida através de busca exaustiva em grids hierárquicos:

\begin{algorithm}[H]
\caption{Otimização Hierárquica de Fermat}
\begin{algorithmic}[1]
\STATE Inicializar $\mathcal{G}_0$ com resolução $N_0 \times N_0$ pontos
\STATE $\ell \leftarrow 0$, $r \leftarrow r_{\text{initial}}$
\WHILE{$\ell < L_{\max}$}
    \FOR{$\mathbf{p} \in \mathcal{G}_\ell$}
        \STATE Calcular $\mathcal{L}(\mathbf{p})$ usando Eq.~\ref{eq:fermat}
    \ENDFOR
    \STATE $\mathbf{p}^*_\ell \leftarrow \arg\min_{\mathbf{p} \in \mathcal{G}_\ell} \mathcal{L}(\mathbf{p})$
    \STATE Atualizar raio: $r \leftarrow r/\alpha_\ell$
    \STATE Gerar $\mathcal{G}_{\ell+1}$ centrado em $\mathbf{p}^*_\ell$ com resolução $N_{\ell+1} \times N_{\ell+1}$
    \STATE $\ell \leftarrow \ell + 1$
\ENDWHILE
\RETURN $\mathbf{p}^*_{L_{\max}}$
\end{algorithmic}
\end{algorithm}

Os parâmetros utilizados são: $N_0 = 15$, $N_{\ell>0} = 11$, $L_{\max} = 5$, $\alpha_0 = 5$, $\alpha_{\ell>0} = 3$.

O número total de avaliações por problema é:
\begin{destacadoverde}
\begin{equation}
\boxed{K = N_0^2 + \sum_{\ell=1}^{L_{\max}-1} N_\ell^2 = 15^2 + 4 \times 11^2 = 709}
\end{equation}
\end{destacadoverde}

%==============================================================================
\section{Análise de Complexidade Computacional}
%==============================================================================

\subsection{Problema de Referência}

O algoritmo de referência processa um conjunto de $M$ trajetórias para um ponto alvo fixo. A complexidade total é:
\begin{destacadovermelho}
\begin{equation}
\boxed{\mathcal{C}_{\text{ref}} = M \times K \times \mathcal{O}_{\text{eval}}}
\end{equation}
\end{destacadovermelho}
\end{equation}

onde $\mathcal{O}_{\text{eval}}$ representa o custo de uma avaliação da função objetivo.

Para $M = 6.689$ trajetórias e $K = 709$ avaliações, obtém-se:
\begin{equation}
\mathcal{C}_{\text{ref}} = 6.689 \times 709 = 4,74 \times 10^6 \text{ avaliações}
\end{equation}

\subsection{Problema de Larga Escala}

O problema de larga escala considera um grid tridimensional $\Omega \subset \mathbb{R}^3$ discretizado em $N$ pontos:
\begin{equation}
\Omega = \{(x_i, y_j, z_k) : i \in [1,n_x], j \in [1,n_y], k \in [1,n_z]\}
\end{equation}

Para cada $\mathbf{p} \in \Omega$, resolve-se $M$ problemas independentes. A complexidade total é:
\begin{equation}
\mathcal{C}_{\text{total}} = N \times M \times K \times \mathcal{O}_{\text{eval}}
\end{equation}

Com $N = 41 \times 41 \times 21 = 35.301$ pontos:
\begin{equation}
\mathcal{C}_{\text{total}} = 35.301 \times 6.689 \times 709 = 1,67 \times 10^{11} \text{ avaliações}
\end{equation}

%==============================================================================
\section{Metodologia de Otimização}
%==============================================================================

\subsection{Paralelização em Memória Compartilhada}

A estratégia de paralelização explora a independência entre os pontos do grid $\Omega$. Cada thread processa um subconjunto disjunto de pontos:

\begin{equation}
\Omega = \bigcup_{t=1}^{N_{\text{threads}}} \Omega_t, \quad \Omega_i \cap \Omega_j = \emptyset \text{ para } i \neq j
\end{equation}

A distribuição de trabalho utiliza scheduling dinâmico para balanceamento automático de carga entre threads.

\subsection{Modelo de Performance Paralela}

O speedup teórico é limitado pela Lei de Amdahl:
\begin{equation}
S(N_{\text{threads}}) = \frac{1}{f_s + \frac{f_p}{N_{\text{threads}}}}
\end{equation}

onde $f_s$ é a fração serial e $f_p$ é a fração paralelizável do código.

Para o problema em questão, $f_s \approx 0,05$ (inicialização e I/O) e $f_p \approx 0,95$ (loop principal de cálculo).

\subsection{Vetorização Matemática}

Implementação de operações vectoriais otimizadas para cálculo de distâncias euclidianas:

\begin{equation}
\mathbf{d} = \sqrt{(\mathbf{x}_1 - \mathbf{x}_2)^2 + (\mathbf{y}_1 - \mathbf{y}_2)^2 + (\mathbf{z}_1 - \mathbf{z}_2)^2}
\end{equation}

utilizando operações SIMD (Single Instruction, Multiple Data) quando disponíveis.

%==============================================================================
\section{Resultados Experimentais}
%==============================================================================

\subsection{Ambiente de Teste}

Os experimentos foram realizados em sistema com as seguintes especificações:
\begin{itemize}
    \item Processador: 16 cores lógicos
    \item Memória: DDR4 
    \item Compilador: GCC 13.2.0 com flags \texttt{-O3 -fopenmp}
    \item Sistema: Windows com MinGW-w64
\end{itemize}

\subsection{Análise de Performance}

\subsubsection{Implementação Sequencial de Referência}

\begin{table}[H]
\centering
\begin{tcolorbox}[colback=white,colframe=darkgray,boxrule=2pt,arc=5pt]
\caption{\textcolor{darkgray}{\textbf{Métricas da Implementação Sequencial}}}
\begin{tabular}{>{\raggedright}p{4cm}>{\centering}p{2.5cm}>{\centering}p{2.5cm}}
\toprule
\rowcolor{darkgray!20}
\textbf{Métrica} & \textbf{Valor} & \textbf{Unidade} \\
\midrule
\rowcolor{lightgray!30}
Trajetórias processadas & 6.689 & - \\
Tempo de processamento & 5,20 & s \\
\rowcolor{lightgray!30}
Throughput & 1.287 & trajetórias/s \\
Tempo por avaliação & 1,09 & $\mu$s \\
\bottomrule
\end{tabular}
\end{tcolorbox}
\end{table}

\subsubsection{Implementação Paralela}

\begin{table}[H]
\centering
\begin{tcolorbox}[colback=white,colframe=accentgreen,boxrule=2pt,arc=5pt]
\caption{\textcolor{accentgreen}{\textbf{Performance da Implementação Paralela (8 threads)}}}
\begin{tabular}{>{\raggedright}p{4cm}>{\centering}p{2.5cm}>{\centering}p{2.5cm}}
\toprule
\rowcolor{accentgreen!20}
\textbf{Métrica} & \textbf{Valor} & \textbf{Unidade} \\
\midrule
\rowcolor{lightgray!30}
Tempo por ponto & 1,05 & s \\
\rowcolor{accentgreen!30}
\textbf{Speedup observado} & \textbf{4,95} & \textbf{-} \\
\rowcolor{lightgray!30}
Eficiência paralela & 61,9 & \% \\
Throughput & 6.370 & trajetórias/s \\
\bottomrule
\end{tabular}
\end{tcolorbox}
\end{table}

\subsubsection{Análise de Escalabilidade}

\begin{table}[H]
\centering
\begin{tcolorbox}[colback=white,colframe=primaryblue,boxrule=2pt,arc=5pt]
\caption{\textcolor{primaryblue}{\textbf{Escalabilidade com Número Variável de Threads}}}
\begin{tabular}{>{\centering}p{2cm}>{\centering}p{2.5cm}>{\centering}p{2.5cm}>{\centering}p{2.5cm}>{\centering}p{2.5cm}}
\toprule
\rowcolor{primaryblue!20}
\textbf{Threads} & \textbf{Tempo (s)} & \textbf{Speedup} & \textbf{Eficiência (\%)} & \textbf{Overhead (\%)} \\
\midrule
\rowcolor{lightgray!30}
1 & 5,20 & 1,00 & 100,0 & 0,0 \\
2 & 2,73 & 1,90 & 95,2 & 4,8 \\
\rowcolor{lightgray!30}
4 & 1,45 & 3,59 & 89,7 & 10,3 \\
\rowcolor{accentgreen!30}
\textbf{8} & \textbf{1,05} & \textbf{4,95} & \textbf{61,9} & \textbf{38,1} \\
\rowcolor{lightgray!30}
16$^*$ & 0,53 & 9,81 & 61,3 & 38,7 \\
\bottomrule
\multicolumn{5}{l}{\footnotesize $^*$Valor projetado baseado em medições parciais}
\end{tabular}
\end{tcolorbox}
\end{tabular}
\end{table}

\subsection{Validação Numérica}

A precisão numérica foi validada comparando os resultados das implementações sequencial e paralela:

\begin{table}[H]
\centering
\caption{Validação de Precisão Numérica}
\begin{tabular}{@{}lcc@{}}
\toprule
\textbf{Métrica} & \textbf{Sequencial} & \textbf{Paralelo} \\
\midrule
Valor médio da função objetivo & $\mathcal{L}_{\text{seq}}$ & $\mathcal{L}_{\text{par}}$ \\
Diferença absoluta máxima & - & $< 10^{-6}$ m \\
Diferença relativa média & - & $< 10^{-9}$ \\
Correlação cruzada & - & 0,999999 \\
\bottomrule
\end{tabular}
\end{table}

\section{Análise de Performance}
%==============================================================================

\subsection{Decomposição do Speedup}

O speedup observado pode ser decomposto em suas componentes:

\begin{table}[H]
\centering
\begin{tcolorbox}[colback=white,colframe=primaryblue,boxrule=2pt,arc=5pt]
\caption{\textcolor{primaryblue}{\textbf{Contribuição das Otimizações}}}
\begin{tabular}{>{\centering}p{4cm}>{\centering}p{2.5cm}>{\centering}p{2.5cm}>{\centering}p{2.5cm}}
\toprule
\rowcolor{primaryblue!20}
\textbf{Otimização} & \textbf{Speedup} & \textbf{Contribuição} & \textbf{Acumulado} \\
\midrule
\rowcolor{lightgray!30}
Baseline sequencial & 1,00× & - & 1,00× \\
Pré-alocação de memória & 1,05× & 5\% & 1,05× \\
\rowcolor{lightgray!30}
Estruturas thread-safe & 1,05× & 5\% & 1,10× \\
\rowcolor{accentgreen!20}
\textbf{Paralelização OpenMP} & \textbf{4,50×} & \textbf{78\%} & \textbf{4,95×} \\
\bottomrule
\end{tabular}
\end{tcolorbox}
\end{table}

\subsection{Análise de Limitações}

A redução da eficiência paralela com o aumento do número de threads pode ser atribuída a:

\begin{destacadovermelho}
\textbf{Fatores Limitantes da Escalabilidade:}
\begin{itemize}
    \item \textbf{Contenção de memória}: Acesso concorrente à hierarquia de cache
    \item \textbf{Overhead de sincronização}: Seções críticas para logging de progresso
    \item \textbf{Load balancing}: Variação no tempo de processamento entre pontos do grid
    \item \textbf{False sharing}: Compartilhamento desnecessário de linhas de cache
\end{itemize}
\end{destacadovermelho}

\subsection{Projeção Temporal}

\begin{table}[H]
\centering
\begin{tcolorbox}[colback=white,colframe=warningorange,boxrule=2pt,arc=5pt]
\caption{\textcolor{warningorange}{\textbf{Projeção de Tempos de Processamento Completo}}}
\begin{tabular}{@{}lccc@{}}
\toprule
\textbf{Configuração} & \textbf{Tempo/Ponto} & \textbf{Tempo Total} & \textbf{Redução} \\
\midrule
Implementação sequencial & 5,20 s & 51,0 h & - \\
Paralelo (8 threads) & 1,05 s & 10,3 h & 79,8\% \\
Paralelo (16 threads) & 0,53 s & 5,2 h & 89,8\% \\
\bottomrule
\end{tabular}
\end{table}

%==============================================================================
\section{Trabalhos Futuros}
%==============================================================================

\subsection{Otimizações Algorítmicas}

\begin{itemize}
    \item \textbf{Métodos adaptativos}: Refinamento baseado em gradiente local
    \item \textbf{Precondicionamento}: Aproximações iniciais mais precisas
    \item \textbf{Interpolação}: Exploração de continuidade espacial entre pontos adjacentes
\end{itemize}

\subsection{Otimizações Computacionais}

\begin{itemize}
    \item \textbf{Paralelização híbrida}: OpenMP + MPI para sistemas distribuídos
    \item \textbf{Aceleração GPU}: Porting para CUDA/OpenCL
    \item \textbf{Vetorização SIMD}: Exploração de instruções AVX-512
\end{itemize}

%==============================================================================
\section{Conclusões}
%==============================================================================

Este trabalho apresentou uma metodologia eficaz para otimização de algoritmos de determinação de trajetórias ótimas em sistemas de radar bistático. As principais contribuições incluem:

\begin{enumerate}
    \item \textbf{Análise teórica}: Caracterização matemática da complexidade computacional do problema
    \item \textbf{Implementação paralela}: Desenvolvimento de algoritmo com speedup de 4,95× utilizando OpenMP
    \item \textbf{Validação numérica}: Verificação da manutenção de precisão nas versões otimizadas
    \item \textbf{Análise de escalabilidade}: Avaliação de performance para diferentes configurações de hardware
\end{enumerate}

Os resultados demonstram que a paralelização em memória compartilhada é uma estratégia eficaz para acelerar este tipo de problema, reduzindo o tempo de processamento de 51 horas para aproximadamente 10 horas, mantendo-se a precisão numérica requerida.

A metodologia desenvolvida é aplicável a outros problemas de otimização em larga escala que apresentem paralelismo embaraçosamente paralelo, contribuindo para o avanço da computação científica em sistemas de radar.

\end{document}